% EE225B Final Project
\documentclass[11pt,twocolumn]{article}
\title{\textbf{Improved Kannada Text Detection\\ in Natural Scene Images}}
\author{Achal~Dave,~Gautam~Gunjala}
\date{}

\begin{document}
\maketitle
Abstract: 

\section{Introduction}

\section{Algorithm}

This is a modified version of the Stroke Width Transform. Currently, the modifications should improve detections on any language, but our goal is to improve the accuracy with which Kannada text is detected and to decrease the false positive rate, even at the cost of detecting other languages.

\subsection{Implemented modifications}

Modifications that we've implemented so far.

\subsection{Erosion (morphological)}

Morphological erosion is used for two improvements: removal of small non-axis aligned line-like components, and stroke width variance thresholding.

The original SWT thresholds components using a few metrics, primarily ones involving their height, width, and stroke width variance. Unfortunately, this results in a number of diagonal lines and small patches. Eroding the component and thresholding by the number of pixels fixes this issue.

Second, in natural images, letters are often not perfectly segmented by most edge detectors (especially for signs in developing countries that are often carved in stone), which results in components containing small erroneous pixels outside of letters. The regular stroke width variance thresholding ignores these due to the extra pixels; to avoid this, we use an erosion to only look at the variance of the core of the image.

\subsection{Multi step thresholding}

This is more of an implementation detail, but it's extremely helpful for performance to threshold in more than one step. We found that the default SWT provided millions of extremely small components (1-2 pixels). Unfortunately, the various thresholding methods in this algorithm are difficult to vectorize, meaning it is necessary to loop over every component. However, we can vectorize the removal of very small components using typical MATLAB (or any other languages') functions, reducing the components to thousands before doing further checks.

\subsection{Proposed modifications}

\subsection{Gradient histograms}

Kannada characters are especially circular in nature, a feature that we should be able to exploit in detection. We attempted to look at a ray histogram of gradients for each connected component, but found it unstable in our first few tries.

\subsection{Surrounding text voting}

Text tends to be surrounded by other text. Using a dual threshold, we can eliminate components that are obviously incorrect (by some measure), and vote on components that we are less certain for by looking at their surroundings (a la the Canny edge detector).

\subsection{Color quantization}

Ikica and Peer discuss a modified color reduction method using SWT voting. It may be possible to invert this and use color reduction to vote on SWT components,, but we have not looked into this much.


\section{Conclusion}



The conclusion goes here.

\begin{thebibliography}{1}

\bibitem{IEEEhowto:kopka}
H.~Kopka and P.~W. Daly, \emph{A Guide to {\LaTeX}}, 3rd~ed. Harlow, England: Addison-Wesley, 1999.

\end{thebibliography}

\end{document}