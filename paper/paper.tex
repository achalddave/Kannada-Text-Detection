% EE225B Final Project
\documentclass[11pt]{article}

\usepackage[margin={2cm,2cm}]{geometry}
\usepackage{multicol}
\usepackage{parskip}
\usepackage{graphicx}
\usepackage{capt-of}
\setlength{\columnsep}{1cm}

\setlength{\parindent}{0pt}

\title{\textbf{Improved Text Detection\\ in Natural Scene Images}}
\author{Achal~Dave,~Gautam~Gunjala}
\date{}

\begin{document}
\maketitle

\begin{multicols}{2}

\section{Abstract}

\textit{We present several modifications to the Stroke Width Transform (SWT), a
local image operation that uses geometric characteristics of natural scene
images to identify text within them. We add several novel checks that make the
text detection algorithm more robust, lowering the rate of false positives
among image components flagged with negligible detriment to its running time
with respect to existing implementations. Extensive testing performed on images
from Bangalore, Karnataka, indicates that the improved algorithm can be
reliably used to detect and extract both English and Kannada text from the
images.}

\section{Introduction}
Text detection in natural scene images is an important application of image
processing and computer vision. With the exponential growth of images on the
internet, it's become more and more interesting to find ways to classify and
understand images automatically. Text is a significant piece of the solution,
as it contains key information about both the context and location of an image.
Unfortunately, most past text detection and recognition work focuses on a few
languages, such as English (and others with a Latin script) and Chinese. In
this paper, we focus on Kannada detection, although our modifications
generalize to most languages. In future work, we hope to work on recognizing
Kannada characters.

We present an algorithm based on the Stroke Width Transform that provides
improved text detection in natural images. We implement two distinct
classifiers that utilize the output of the Stroke Width Transform (without any
thresholding) along with features extracted by our algorithm, and compare their
results qualitatively. Our original objective was to improve text detection on
the Kannada language, although our modifications generalized to most languages.

Unfortunately, due to the nature of Kannada, a language in which consonants can
be joined arbitrarily, it is difficult to use many typical algorithms that rely
on character-based classifiers for detection and recognition. We are unaware of
any published Kannada text that performs even reasonably well, which makes the
problem of detection much more interesting. In future work, we hope to revisit
recognition using the output of our detection algorithm as presented in this
paper.

\section{Dataset}

We perform our algorithms on the Chars74k dataset gathered by Microsoft
Research. The dataset consists of natural images with Kannada and English text.
As we originally were focusing on Kannada text, we were constrained to a small
dataset; specifically, Chars74k is the only publicly available dataset we were
able to find that contains reasonably labeled characters in natural images.
Unfortunately, the dataset does not guarantee that all occurrences of text are
labelled, making it difficult to quantize error precisely.

\section{Feature Extraction}

Our algorithm begins with a modified version of the Stroke Width Transform that
extracts connected components in the manner described in the next section. We
then discuss additional operations that we perform on each extracted component
that help to better characterize the component as text or not.

\subsection{Stroke Width Transform}

The Stroke Width Transform (SWT) relies on the key insight that text in natural
images consists of lines of generally uniform thickness. Using this assumption,
it is possible to extract components whose stroke widths, defined as the
shortest distance between edges, are nearly uniform. A detailed explanation of
the transform is provided in the original paper; we provide a brief overview
here.

The Stroke Width Transform works as follows:

\begin{itemize}

  \item Detect all edges in an image, using a high pass filter (Sobel, Prewitt)
        or edge detection algorithm (Canny). It is crucial that in this step, each
        point in the image is classified as an edge or not an edge.

  \item Create a new image ($I_{swt}$) of the same size as the original. If a pixel
        is found to be text, the corresponding pixel in this new image will hold the
        stroke width of the line containing it. Initialize each value in this new
        image to value +Inf

  \item Consider only the subset of edge pixels in the original image. For each
        point in this subset:
        \begin{itemize}
            \item Walk in the direction of its gradient (calculated with
                  respect to the original image)

            \item Continue until another edge pixel with a gradient in the
                  opposite direction is encountered, and then terminate.

            \item Set \texttt{curr\_width} to the Euclidean distance between the
                 start and end pixel.

            \item For each pixel \texttt{(x,y)} walked over in this process, set the
                  stroke width to be $I_{swt}(x,y) =$
                  \texttt{min(}$I_{swt}(x,y)$\texttt{, curr\_width)}
          \end{itemize}

    \item Next, we look for connected components in $I_{swt}$. For each pair
    of neighboring pixels, we create an edge if the stroke widths are off by a
    factor of 3.

    \item Finally, the SWT thresholds these components on the variance of the
    stroke widths, and the dimensions of the components.

\end{itemize}

{
\centering
\includegraphics[width=0.5\textwidth]{assets/swt_stroke}
\captionof{figure}{Image taken from the original SWT paper. (a) An example of a
stroke. (b) \textbf{p} is a pixel in the edge image; by walking along the
gradient, we encounter \textbf{q}. (c) all pixels on the path are assigned the
same stroke width.}
}

\subsection{Modifications}

Although the features described in the Stroke Width Transform are useful, they
do not provide enough information to easily distinguish text components from
non-text components. The SWT, as it is currently implemented, ignores a number
of key aspects of text components, resulting in many noisy detections. In the
following sections, we discuss modifications to the Stroke Width Transform that
attempt to overcome these issues. Although these added procedures result in
additional running time � for example, one must be performed serially - the
change is negligible and the improvement in the accuracy of the results is
worth the additional time taken.

\subsection{Morphological Erosion}

The SWT does basic thresholding based on the bounding box of each component.
However, small, noisy components are still allowed. Components that consist of
thin diagonal lines, for example, display proper bounding boxes but are
generally undesirable.

{
\centering
\includegraphics[width=0.5\textwidth]{assets/swt_diag}
\captionof{figure}{Diagonal lines detected from SWT algorithm}
\label{fig:swt_diag}
}

For example, Figure~\ref{fig:swt_diag} demonstrates such diagonal lines which
may be identified as parts of text by the original SWT algorithm.  To account
for such instances, we thin each component using a morphological erosion, and
remove any component that disappears following the erosion. As
Figure~\ref{fig:swt_diag_morphed} shows, the morphological erosion technique
used in our modification of the SWT was able to successfully dismiss the
diagonal lines as not text.

{
\centering
\includegraphics[width=0.5\textwidth]{assets/swt_diag_morphed}
\captionof{figure}{Morphological erosion applied to Figure~\ref{fig:swt_diag}}
\label{fig:swt_diag_morphed}
}

\subsection{Quantization error}

Components containing text tend to have smooth surroundings. In natural scene
images, we tend to see dark text on a light background or vice versa, with a
two-color scheme, be it black and white or some other combination. On the other
hand, most noisy components are surrounded by pixels with high variance. To
exploit this fact, we convert the pixels in each component to a binary image,
such that we have only the component on a plain background of opposite color.
We then extract the corresponding window from the original image and binarize
that. We claim that if the component is text, then the two binary images should
have a relatively low root mean squared error (RMS) normalized by window size.
When this error is high, the component is unlikely to be text. This further
filtering of components by quantization also contributes to the differences
between the two images in the previous subsection.

\section{Classification}

We implemented two separate methods for classifying the components output by
the SWT: thresholding and decision trees, and compare their results.

\subsection{Threshold and Vote}

Our first method follows the Stroke Width Transform in hand tuning thresholds
for each feature. While our features improve the performance of detections over
the raw SWT with just this thresholding, we still see a number of false
positives in various parts of the image. Due to the instability of these
thresholds, attempting to remove these false positives by tighter values
results in missing text detections.

To overcome the instability of the thresholds, we turn to a hysteresis method
similar to that used by the Canny edge detector. We maintain two sets of
thresholds: a strict threshold that only admits text, and a loose threshold
that admits many components, whether text or not.

We note that text in natural images is often surrounded by other text, and we
rarely see singleton characters floating in an image. Consequently, we use a
method similar to kernel density estimation to calculate the probability of
text located in a certain region in the image. We center Gaussians at the
centers of the bounding boxes of components that pass the strict threshold, and
create a probability map containing the likelihoods of text at various
locations in the image.

{ \centering
\includegraphics[width=0.5\textwidth]{assets/18th_cross}
\captionof{figure}{Grayscale image of a sign}
\label{fig:18th_cross} 
}

We use these probabilities to vote on components that only passed the loose
threshold. Note that in the probability map, we see areas outside of the text
location that have peaks, as it is still difficult to tune the strict threshold
well. Due to these troubles, despite the improvements over the Stroke Width
Transform, we move on to a decision tree based classifier.

\subsection{Decision Trees}
In order to avoid issues with tuning, we turn to a simple decision tree based
classifier. Decision trees are well suited to this problem for two reasons: the
problem is definitely not linearly separable, as these features simply are not
intricate enough, and because they can easily model different relationships
between features based on thresholds. We attempted to train an SVM on a small
set of data as part of a quick experiment, but found that overall decision
trees were both theoretically more sound and performed better.

{ \centering
\includegraphics[height=0.5\textwidth]{assets/bangalore}
\captionof{figure}{Street sign with text.}
\label{fig:bangalore} }

{ \centering
\includegraphics[width=0.5\textwidth]{assets/bangalore_missed}
\captionof{figure}{Components classified by a decision tree as text. Although
the number of false positives is low, we still miss one or two text
detections.}
\label{fig:bangalore_missed} }

We use the same features from earlier, without the probability density map, as
deriving the probability densities requires a threshold in the first place.
Specifically, our feature vector contains the following: component�s raw
bounding box size, component's bounding box size proportional to the image, the
mean and variance of the stroke width, the quantization error, and the number
of pixels left after a morphological erosion.

We extracted 84 labelled images with text, containing approximately 1000 text
components, and 5000 non-text components, all of which pass very coarse
thresholds from the Stroke Width Transform (precisely, ones that are more than
1 pixel wide and less than half the size of the entire image). Using 10-fold
cross validation, we were able to get an average accuracy of approximately 17\%.

{
\centering
\includegraphics[width=0.4\textwidth]{assets/green}
\captionof{figure}{Another street sign.}
\label{fig:green}
}

{
\centering
\includegraphics[height=0.3\textwidth]{assets/green_rejected}
\captionof{figure}{Components rejected by our decision tree.}
\label{fig:green_rejected}
}

Note, however, that these results are preliminary in two important ways. First,
the dataset does not guarantee that all text is labelled, just that all
labelled text is guaranteed to be text. This means not only that many text
components are tossed in as negative training data, but also that our errors
may be exaggerated at test time. Second, we have a small dataset and a large
number of negative data, which means it�s hard to make decisive conclusions
about this performance.

From both qualitative and rough quantitative measures, however, our decision
tree outperforms both the probability density method and the SWT method
implementations in our experiments.

\subsection{Conclusion}

We have shown a number of modifications to the Stroke Width Transform that
improve its false positive error while attempting to maintain the true
positives. These perform reasonably well, although, since we are focusing on
improving text detection specifically for the Kannada language, future efforts
can be made in exploiting features characteristic of the Kannada script.

\begin{thebibliography}{1}

\bibitem{SWT}
Boris Epshtein, Eyal Ofek, Yonatan Wexler. \emph{Detecting Text in Natural
Scenes with Stroke Width Transform}.

\bibitem{char74k}
T. E. de Campos, B. R. Babu and M. Varma. Character recognition in natural
images. In \emph{Proceedings of the International Conference on Computer Vision
Theory and Applications (VISAPP), Lisbon, Portugal}, February 2009.

\bibitem{KannadaCharRecog}
M. Mahadeva Prasad, M. Sukumar, A. G. Ramakrishnan. \emph{Divide and Conquer
Technique in Online Handwritten Kannada Character Recognition.}

\bibitem{SWTvoting}
Andrej Icica, Peter Peer. \emph{SWT Voting-Based Color Reduction For Text Detection
in Natural Scene Images.}

\bibitem{GithubImpl}
DetectText, An implementation of the SWT in C++. https://github.com/aperrau/DetectText

\end{thebibliography}

\end{multicols}

\end{document}
